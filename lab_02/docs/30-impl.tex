\chapter{Технологическая часть}

\section{Требования к ПО}
К программе предъявляется ряд требований:
\begin{itemize}
    \item предоставить возможность работы в двух режимах: проведения эксперимента и ручного тестирования;
    \item в режиме ручного тестирования предоставить пользователю как самостоятельно осуществлять ввод двух целочисленных матриц, которые 
    необходимо перемножить, так и использовать автозаполнение матриц заданных размерностей псевдослучайными числами от 0 до 99;
	\item выходными данными программы в ручном режиме являются три матрицы - результаты перемножения матриц c исользованием классического алгоритма, алгоритма
    Винограда и оптимизированного алгоритма Винограда;
    \item в режиме проведения эксперимента выходными данными является результат эксперимента в текстовом виде, а также графики в виде файла с изображением.
\end{itemize}

\section{Средства реализации}
Для реализации алгоритмов перемножения матриц был выбран язык программирования Python3\cite{python}, что обусловлено простотой 
и скоростью написания программ, а также наличием встроенных библиотек для построения графиков функций и тестирования. 
В качестве среды разработки был выбран \texttt{PyCharm}\cite{pycharm}, как наиболее популярная IDE для Python3. 

\section{Функциональное тестирование}

При разработке функционалых тестов были выделены следующие классы эквивалентности:

\begin{itemize}
    \item хотя бы один из размеров матрицы неположительный;
    \item кол-во столбцов первой матрицы не совпадает с кол-вом строк второй;
    \item хотя бы одна матрица вырождена в вектор-строку или вектор-столбец;
    \item обе матрицы размером 1 х 1;
    \item обе матрицы являются квадртаными с четным размером;
    \item обе матрицы являются квадртаными с нечетным размером;
    \item произвольные прямоугольные матрицы корректных размеров.
\end{itemize}

В соответствии с данными классами эквивалентности были разработаны тесты, представленные в таблице \ref{tab:tests}.

\begin{table}[h!]
	\begin{center}
        \captionof{table}{Тестирование функций}
		\begin{tabular}{|c|c|c|c|c|}
			\hline
			Размеры М1 & Размеры М2 & М1 & М2 & Результат \\ 
            \hline
            $-1$ x $2$ &
            --- &
			--- &
			--- &
			Ошибка \\
            \hline
            $2$ x $3$ &
            $2$ x $2$ &
			--- &
			--- &
            Ошибка \\
            \hline
            $1$ x $2$ &
            $2$ x $2$ &
			$\begin{pmatrix}
                1 & 1
			\end{pmatrix}$ & 
            $\begin{pmatrix}
                1 & 1\\
                1 & 1
			\end{pmatrix}$ & 
            $\begin{pmatrix}
                2 & 2
			\end{pmatrix}$ \\ 
            \hline
            $1$ x $1$ &
            $1$ x $1$ &
			$\begin{pmatrix}
                2
			\end{pmatrix}$ & 
            $\begin{pmatrix}
                3
			\end{pmatrix}$ & 
            $\begin{pmatrix}
               6
			\end{pmatrix}$ \\ 
            \hline
            $2$ x $2$ &
            $2$ x $2$ &
			$\begin{pmatrix}
			1 & 1\\
			1 & 1
			\end{pmatrix}$ &
			$\begin{pmatrix}
			1 & 1\\
			1 & 1
			\end{pmatrix}$ &
			$\begin{pmatrix}
			2 & 2\\
			2 & 2\\
			\end{pmatrix}$\\
            \hline
            $3$ x $3$ &
            $3$ x $3$ &
			$\begin{pmatrix}
			1 & 1 & 1\\
			1 & 1 & 1\\
			1 & 1 & 1
			\end{pmatrix}$ &
			$\begin{pmatrix}
			1 & 1 & 1\\
			1 & 1 & 1\\
			1 & 1 & 1
			\end{pmatrix}$ &
			$\begin{pmatrix}
            3 & 3 & 3\\
            3 & 3 & 3\\
            3 & 3 & 3
			\end{pmatrix}$\\
            \hline
	        $2$ x $3$ &
            $3$ x $2$ &
			$\begin{pmatrix}
			1 & 1 & 1\\
			1 & 1 & 1
			\end{pmatrix}$ &
			$\begin{pmatrix}
			1 & 1 \\
			1 & 1 \\
			1 & 1 
			\end{pmatrix}$ &
			$\begin{pmatrix}
            3 & 3\\
            3 & 3
            \end{pmatrix}$\\
            \hline
        \end{tabular}
        \label{tab:tests}
	\end{center}
\end{table}

\section{Реализация алгоритмов}

Реализации классического алгоритма перемножения матриц, алгоритма Винограда, а также оптимизированного 
алгоритма Винограда соответственно представлены в листингах \ref{lst:classic}--\ref{lst:win-imp}.



\section{Вывод}
В данном разделе были выделены классы эквивалентности для операции перемножения матриц, 
на основе которых разработаны функциональные тесты для ПО, также были реализованы сами алгоритмы 
перемножения матриц на языке Python3.