\chapter{Аналитическая часть}

В данной лабораторной работе многопоточность изучается на при­мере 
алгоритма поворота в пространстве фигуры, представленной в виде массива точек;

Эта задача является весьма актуальной, т.к. компьютерная графи­ка
стала неотъемлемой частью повседневной интернет-жизни человека, и
существует потребность в быстром рендеринге изображения, например,
при анимации поворота фигуры.

Как известно, в экранной плоскости изображение представляет
из себя набор пикселей (точек). Очевидно, что какая-либо фигура - это
тоже набор точек экранной плоскости, и для поворота фигуры требуется
над каждой ее точкой произвести преобразование для получения новой
позиции.

Если использовать один поток для рендеринга изображения, то
при большом количестве и сложности фигур изображение будет 
генери­роваться ощутимо долго, что будет приносить человеку дискомфорт при
восприятии, однако если распараллелить этот процесс, т.е. параллельно
генерировать части изображения (т.к. эта операция выполняется
 неза­висимо для каждой точки), то это может дать коллосальной прирост
производительности.

Перейдем к рассмотрению алгоритма поворота в пространстве\cite{alg}

\section{Алгоритм поворота точек в пространстве}

Любое вращение в трёхмерном пространстве может быть представлено 
как композиция поворотов вокруг трёх ортогональных осей 
(например, вокруг осей декартовых координат). Этой композиции 
соответствует матрица, равная произведению соответствующих трёх
 матриц поворота \cite{math}


Матрицами вращения вокруг оси декартовой системы координат на угол 
$\alpha$ в трёхмерном пространстве с неподвижной системой координат 
являются:

Вращение вокруг оси $x$ (\ref{eq:mx})
\begin{equation}
	\label{eq:mx}
	M_{x}(\alpha) = \begin{pmatrix}
		1 & 0  & 0\\
		0 & \cos\alpha  & -\sin\alpha\\
		0 & \sin\alpha  & \cos\alpha
	\end{pmatrix},
\end{equation}

Вращение вокруг оси $y$ (\ref{eq:my})

\begin{equation}
	\label{eq:my}
	M_{y}(\alpha) = \begin{pmatrix}
		\cos\alpha & 0  & \sin\alpha\\
		0 & 1  & 0\\
		-\sin\alpha & 0 & \cos\alpha
	\end{pmatrix},
\end{equation}


Вращение вокруг оси $z$ (\ref{eq:mz})
\begin{equation}
	\label{eq:mz}
	M_{z}(\alpha) = \begin{pmatrix}
		\cos\alpha & -\sin\alpha  & 0\\
		\sin\alpha & \cos\alpha  & 0\\
		0 & 0 & 1
	\end{pmatrix},
\end{equation}

Положительным углам при этом соответствует вращение вектора против 
часовой стрелки в правой системе координат, и по часовой стрелке в 
левой системе координат, если смотреть против направления 
соответствующей оси[2]. Например, при повороте на угол 
$\alpha=90^{\circ }$ вокруг оси $z$ ось $x$ переходит в $y$:
$M_{z}(90^{\circ })\cdot \mathbf{e}_{x}=\mathbf{e}_{y}$. 
Аналогично, $M_{y}(90^{\circ })\cdot \mathbf{e}_{z}=\mathbf{e}_{x}$ 
и $M_{x}(90^{\circ })\cdot \mathbf{e}_{y}=\mathbf{e}_{z}$. 

Пусть необходимо повернуть точку $P(x, y, z)$ вокруг оси $Z$ на угол $\phi$. 
Изображение новой точки обозначим через $P'(x', y', z')$. Координаты новой точки
будут рассчитываться согласно выражениию \ref{eq:newcoords}
\begin{equation}
	\label{eq:newcoords}
	\begin{cases}
		x' = x \cdot \cos\phi - y \cdot \sin\phi\\
		y' = x \cdot \sin\phi + y \cdot \cos\phi\\
		z' = z
	\end{cases}
\end{equation}


Таким образом, распараллеливание будет заключаться в том, что
массив точек будет разбиваться на подмассивы, для каждого из которых
независимо от других будет решаться задача преобразования.

\section{Вывод}

Таким образом, к разрабатываемой программе предъявляются следующие требования:
\begin{enumerate}
	\item Входными данными программы является текстовый файл, содержащий набор точек $x, y, z$, 
	которые необходимо преобразовать и точку $x_{c}, y_{c}, z_{c}$, относительно которой 
	необходимо осуществить поворот, а также углы поворота по каждой из трех осей угол $\phi_{x}, \phi_{y}, \phi_{z}$, на который необходимо повернуть фигуру.
	\item Выходными данными программы также является текстовый файл, содержащий набор преобразованных точек фигуры.
	\item Необходимо предоставить пользователю возможность выбора количества потоков, 
	которые будут использованы выполнения необходимых преобразований, в частности возможность 
	выбрать только один поток выполнения.
	\item Необходимо предоставить возможность работы в двух режимах: проведения эксперимента и ручного тестирования.
	\item Программа должна корректно реагировать на ошибки пользова­теля и невалидные данные
	о фигуре или угле поворота; в таком случае программа должна сообщить пользователю о некорректности
	данных или невозможности преобразования;
\end{enumerate}


