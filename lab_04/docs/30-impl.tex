\chapter{Технологическая часть}
В данном разделе приведны средства реализации, результаты тестирования и листинги кода.

\section{Средства реализации}

Для реализации распараллеливания алгоритма был выбран язык программирования \texttt{C++} \cite{cpp}, 
что обусловлено поддержкой нативных потоков ядра, а также простотой интерфейса библиотеки для взаимодейсвтия с ними. 
В качестве среды разработки был выбран \texttt{CLion}\cite{clion}, как наиболее популярная IDE для \texttt{C++}. 

\section{Функциональное тестирование}

В соответствии с выделеными классами эквивалентности были разработаны тесты, 
представленные в таблице \ref{tab:tests}.
\begin{table}[h!]
	\begin{center}
        \captionof{table}{Тестирование функций}
		\begin{tabular}{|c|c|c|c|c|c|}
			\hline
			Массив точек & Центр & Углы, $^{\circ}$ & Потоки & Ожидаемый рез. & Реальный рез. \\ 
            \hline
            $[(1,0)]$ &
            $(0,0)$ &
            $(0,0,0)$ &
            $1$ &
            $[(1,0)]$ &
            $[(1,0)]$ \\
            \hline
            $[(1,0)]$ &
            $(0,0)$ &
            $(0,0,90)$ &
            $1$ &
            $[(0,1)]$ &
            $[(0,1)]$ \\
            \hline
            $[(1,0)]$ &
            $(0,0)$ &
            $(0,0,-90)$ &
            $1$ &
            $[(-1,0)]$ &
            $[(-1,0)]$ \\
            \hline
            $[ ]$ &
            $(5,2)$ &
            $(1,1,1)$ &
            $1$ &
            $[ ]$ &
            $[ ]$ \\
            \hline
            $[(1,0)]$ &
            $(0,0)$ &
            $(0,0,0)$ &
            $1$ &
            $[(1,0)]$ &
            $[(1,0)]$ \\
            \hline
            $[(1,0), (1,0)]$ &
            $(0,0)$ &
            $(0,0,90)$ &
            $1$ &
            $[(0,1), (0,1)]$ &
            $[(0,1), (0,1)]$ \\
            \hline
            $[(1,0)]$ &
            $(0,0)$ &
            $(0,0,0)$ &
            $1$ &
            $[(1,0)]$ &
            $[(1,0)]$ \\
            \hline
            $[(1,0)]$ &
            $(0,0)$ &
            $(0,0,0)$ &
            $8$ &
            $[(1,0)]$ &
            $[(1,0)]$ \\
            \hline
            $[(1,0)]$ &
            $(0,0)$ &
            $(0,0,0)$ &
            $0$ &
            Ошибка &
            Ошибка \\
            \hline
            $[(q,0)]$ &
            $(2,1)$ &
            $(0,0,0)$ &
            $1$ &
            Ошибка &
            Ошибка \\
            \hline
            $[(0,0)]$ &
            $(1,2)$ &
            $(0,q,0)$ &
            $1$ &
            Ошибка &
            Ошибка \\
            \hline
        \end{tabular}
        \label{tab:tests}
	\end{center}
\end{table}

Все тесты пройдены успешно.

\section{Реализация алгоритмов}

В листинге \ref{lst:rotates} представлена реализация алгоритма поворота точек.
\begin{lstlisting}[label=lst:rotates,caption={Реализация алгоритма поворота точек}]
struct double3 {
    double x;
    double y;
    double z;
};

void rotate_points(std::vector<double3>& points, double3& center, const double3& rot_data, int idx0, int idx1) {
    for (int i = idx0; i <= idx1; ++i) {
        translate_point(points[i], {-center.x, -center.y, -center.z});
        rotate_point(points[i], rot_data);
        translate_point(points[i], {center.x, center.y, center.z});
    }
}

void rotate_point(double3& point, const double3& rot_data) {
    rotate_x_axis(point, rot_data.x);
    rotate_y_axis(point, rot_data.y);
    rotate_z_axis(point, rot_data.z);
}


void translate_point(double3& p, const double3& tr_data) {
    p.x += tr_data.x;
    p.y += tr_data.y;
    p.z += tr_data.z;
}
\end{lstlisting}

В листинге \ref{lst:conc_rotate} представлена реализация алгоритима поворота точки вокруг
каждой из осей.
\begin{lstlisting}[label=lst:conc_rotate,caption={Реализация алгоритма поворота точек}]
void rotate_x_axis(double3& p, const double angle) {
    double cos_theta = cos(to_rad(angle));
    double sin_theta = sin(to_rad(angle));
    double temp_y = p.y;
    p.y = p.y * cos_theta - p.z * sin_theta;
    p.z = temp_y * sin_theta + p.z * cos_theta;
}

static void rotate_y_axis(double3& p, const double angle) {
    double cos_theta = cos(to_rad(angle));
    double sin_theta = sin(to_rad(angle));

    double temp_x = p.x;
    p.x = p.x * cos_theta - p.z * sin_theta;
    p.z = temp_x * sin_theta + p.z * cos_theta;
}

static void rotate_z_axis(double3& p, const double angle) {
    double cos_theta = cos(to_rad(angle));
    double sin_theta = sin(to_rad(angle));

    double temp_x = p.x;
    p.x = p.x * cos_theta - p.y * sin_theta;
    p.y = temp_x * sin_theta + p.y * cos_theta;
}
\end{lstlisting}

В листинге \ref{lst:distr} представлена реализация алгоритма распределения точек по потокам.
\begin{lstlisting}[label=lst:distr,caption={Реализация алгоритма поворота точек}]
void handle(vector<double3>& points, double3& center, const double3& rotate_data, int num_of_threads = 1) {
    int points_per_thread = int(points.size()) / num_of_threads;
    int remaining_data = int(points.size()) - points_per_thread * num_of_threads;
    vector<thread> threads;

    int last = -1, from, to;
    for (int i = 0; i < num_of_threads; ++i) {
        from = last + 1;
        if (i < remaining_data) {
            to = last + points_per_thread + 1;
            last += points_per_thread + 1;
        } else {
            to = last + points_per_thread;
            last += points_per_thread;
        }
        threads.push_back(thread(rotate_points, points, center, rotate_data, from, to));
    }

    for (auto& t: threads)
        t.join();
}
\end{lstlisting}

\section{Вывод}

В данном разделе были реализованы последовательная и распа­раллеленная версии 
алгоритма поворота точек фигуры в пространстве. Данные реализации алгоритмов были
протестированы функциональными тестами, построенными на основе выделенных классов 
эквивалентности.
