\chapter*{Введение}
\addcontentsline{toc}{chapter}{Введение}

Многопоточность - способность центрального процессора или од­ного 
ядра в многоядерном процессоре одновременно выполнять несколь­
ко процессов или потоков, соответствующим образом поддерживаемых 
операционной системой

Другими словами, процесс, порождённый в операционной систе­ме, 
может состоять из нескольких потоков, выполняющихся «параллель­но»,
то есть без предписанного порядка во времени. При выполнении
некоторых задач такое разделение может достичь более эффективного
использования ресурсов вычислительной машины.

\textit{Процесс} - это выполняющаяся программа и все её элементы: 
ад­ресное пространство, глобальные переменные, регистры, стек, открытые
файлы и т.д. \textit{Поток} - это наименьшая последовательность 
запрограм­мированных команд, которые могут управляться планировщиком 
неза­висимо. \textit{Задача} - абстрактная концепция работы, которая должна быть
выполнена.

Многопоточная парадигма стала более популярной с конца 1990-х
годов, поскольку усилия по дальнейшему использованию параллелизма
на уровне инструкций застопорились.


Смысл многопоточности - квазимногозадачность на уровне одного
исполняемого процесса, то есть все потоки выполняются в адресном 
про­странстве процесса. Кроме этого, все потоки процесса имеют не только
общее адресное пространство, но и общие дескрипторы файлов. 
Выпол­няющийся процесс имеет как минимум один (главный) поток.


Преимущества многопоточности:
\begin{itemize}
	\item отзывчивость: один из потоков может обеспечивать быстрый от­
	клик, пока другой поток заблокирон или работает;
	\item разделение ресурсов: поток разделяет код, данные;
	\item экономичность: переключение контекста значительно быстрее,
	чем у процессов;
	\item масштабирование: однопоточные процессоры могут выполняются
	только на одном процессоре, многопоточные в многопроцессорных систе­
	мах - на нескольких процессорах.
\end{itemize}

Недостатки многопоточности:
\begin{itemize}
	\item необходимо заботиться о корректном распараллеливании задачи,
	иначе это может привести к блокировкам и ошибкам доступа;
	\item необходимо заботиться о разделении задачи на подзадачи сбалан­
	сированно, чтобы использовать ресурсы максимально эффективно;
	\item нужна синхронизация потоков, если одна подзадача зависит от
	результата другой;
	\item сложнее тестирование и отладка;
	\item неправильное разделение данных приводит к ошибкам.
\end{itemize}



Целью работы является изучение и реализация принципов многопоточности 
на примере алгоритма поворота фигурыв в пространстве, представленной 
в виде массива точек. 
Для достижения поставленной цели необходимо выполнить следующие задачи:
\begin{itemize}
	\item изучить понятие параллельных вычислений, а также принцип ра­боты 
	алгоритма поворота массива точек в пространстве;
	\item разработать схему рассматриваемого алгоритма в последовательном
	и параллельном вариантах;
	\item описать используемые структуры данных;
	\item реализовать последовательный и параллельный алгоритм поворо­та
	массива точек в пространстве;
	\item протестировать разработанное ПО;
	\item сравнить временные характеристики реализованных версий алго­ритма 
	экспериментально;
	\item на основании проделанной работы сделать выводы.
\end{itemize}




