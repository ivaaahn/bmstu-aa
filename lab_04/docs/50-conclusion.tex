\chapter*{Заключение}
\addcontentsline{toc}{chapter}{Заключение}

В ходе выполнения лабораторной работы была выполнены поставленные задачи, а именно:

\begin{itemize}
	\item рассмотрено и изучено понятие параллельных вычислений, а также принцип ра­боты 
	алгоритма поворота массива точек в пространстве;
	\item разработана схема рассматриваемого алгоритма в последовательном
	и параллельном вариантах;
	\item реализованы последовательный и параллельный алгоритмы поворо­та
	массива точек в пространстве;
	\item выделены классы эквивалентности для данного алгоритма;
	\item на основе выделенных классов эквивалентности 
	разработаны функциональные тесты для программы;
	\item экспериментально проведен сравнительный анализ быстродействия алгоритмов.
\end{itemize}

Экспериментально были подтверждены различия между последо­вательной и 
параллельной реализациями алгоритма:

Из полученных выше результатов замеров процессорного време­ни
можно сделать вывод, что использование многопоточности однознач­но 
может дать прирост эффективности. 

В данной лабора­торной работе в качестве примера рассматривался алгоритм поворота
массива точек фигуры в пространстве, и при использовании много­
поточности удалось добиться как минимум 6-кратного увеличения про­изводительности.

Даже при использовании двух потоков, вместо одонго, производительность увеличилась почти в
2 раза, однако при количестве потоков, равном количеству физических ядер (8 ядер), рост про­изводительности
начинает замедляться, а после количества потоков, равного 16 - столько логических ядер 
имеет процессор, время выполнения алгоритма практически стабильно.

Максимальное количество потоков, при которых выполнялись за­меры - 64
и при этом количестве производительность выросла в 6 раз, 
однако нельзя утверждать, что так будет всегда - если использовать слишком 
много потоков и часть данных для обработки в результате будет малой, то больше времени
бу­дет тратиться на создание и завершение потока, чем на обработку части
информации.

Также можно сделать вывод, что использовать 1 поток неэффек­тивно, 
т.к. по сути это эквивалентно обыкновенному последовательному
алгоритму, однако сверх этого тратятся ресурсы системы на создание и
закрытие потока, и в результате программа работает медленее.

Таким образом, можно сделать вывод, что количество потоков же­
лательно выбирать взвешенно - надо избегать напрасной траты ресурсов
системы, например, не создавать один поток или не создавать слишком
много потоков. Нужно отследить, при каком минимальном количестве
потоков достигается минимально приемлемый выигрыш во времени, т.е.
выбрать пик производительности при возможном минимуме потоков, и
в дальнейшем использовать это число.

Также следует учитывать, что наиболее эффективно распаралле­
ливать анализ больших данных, т.к. маленькие задачи и так выполнятся
достаточно быстро с учетом высокой производительности современных
компьютеров.