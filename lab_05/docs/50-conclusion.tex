\chapter*{Заключение}
\addcontentsline{toc}{chapter}{Заключение}

В ходе выполнения лабораторной работы были выполнены поставленные задачи, а именно:
\begin{itemize}
	\item исследованы основы конвейерных вычислений;
	\item исследованы основные методы организации конвейерных вычислений;
	\item проведено сравнение существующих методов организации конвейерных вычислений; 
	\item приведены схемы рассматриваемых алгоритмов, а именно:
	\begin{itemize}
		\item схема конвейера, содержащего 3 ленты;
		\item схема конвейера, содержащего 3 ленты и реализующего fan--in–fan--out подходы;
	\end{itemize}
	\item описаны использующиеся структуры данных;
	\item описана структура разрабатываемого програмного обеспечения;
	\item определены средства программной реализации;
	\item определены требования к программному обеспечению;
	\item приведены сведения о модулях программы;
	\item проведено тестирование реализованного программного обеспечения;
	\item проведены экспериментальные замеры временных характеристик реализованного конвейера.
\end{itemize}

Прежде чем использовать алгоритмы конвейерных вычислений, необходимо произвести анализ поставленной
задачи, особенно следует учиты­вать следующие факторы:

\begin{enumerate}
	\item Минимальное, максимальное и среднее время простоя заявки в системе.
	\item Минимальное, максимальное и среднее время решения каждого из этапов задачи.
	\item Время переключения между лентами.
	\item Время последовательной передачи управления.
\end{enumerate}

Если затраты на переключение между лентами (диспетчеризация) нивелируются трудоемкостью
задачи, то лучше отдать предпочтение конвейерной реализации. Однако, если задача имеет низкую трудоемкость, 
то, возмонжо, стоит использовать последовательную реализацию, которая будет проще в реализации, что позволит
избежать потенциальных ошибок. Так в случае рассматриваемой в рамках данной работы задачи, конвейерная реализация
алгоритма снизила временные затраты на 30\%, и в некоторых случаях окажется более выгодно использовать чуть менее
эффективный, но более простой и надежный алгоритм.

Также важно найти слабые места в реализации алгоритма - выделить наиболее трудоемкие этапы конвейера, и в том случае, если
есть возможность их распараллелить, то следует использовать подход fan-in / fan-out, что позволит
снизить время ожидания задач в очереди перед данным этапом. Этот подход позволил более чем в 3 
раза снизить временные затраты по сравнению с обычным конвейером. 

