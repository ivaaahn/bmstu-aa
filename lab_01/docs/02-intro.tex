\chapter*{Введение}
\addcontentsline{toc}{chapter}{Введение}

Целью данной лабораторной работы является изучение, реализация и исследование алгоритмов нахождения расстояний Левенштейна и Дамерау – Левенштейна.

Расстояние Левенштейна (редакционное расстояние) — метрика, измеряющая разность между двумя последовательностями символов. Она определяется как минимальное количество односимвольных операций (а именно вставки, удаления, замены), необходимых для превращения одной последовательности символов в другую. Впервые задачу нахождения редакционного расстояния поставил в 1965 году советский математик Владимир Левенштейн при изучении последовательностей 0–1\cite{Levenshtein}.

Расстояние Дамерау—Левенштейна вляется модификацией расстояния Левенштейна: к операциям вставки, удаления и замены символов, определённых в расстоянии Левенштейна добавлена операция транспозиции (перестановки) символов.

Расстояние Левенштейна и его обобщения активно применяются:
\begin{itemize}
	\item для исправления ошибок в слове (в поисковых системах, базах данных, при вводе текста);
	\item для сравнения текстовых файлов утилитой \texttt{diff} и ей подобными;
	\item в биоинформатике для сравнения генов, хромосом и белков.
\end{itemize}


Задачи данной лабораторной работы:
\begin{enumerate}
  	\item изучение алгоритмов нахождения расстояния Левенштейна и Дамерау--Левенштейна;
  	\item применение методов динамического программирования для реализации алгоритмов;
	\item получение практических навыков реализации алгоритмов Левенштейна и Дамерау — Левенштейна;
	\item сравнительный анализ алгоритмов на основе экспериментальных данных;
	\item подготовка отчета по лабораторной работе.
\end{enumerate}