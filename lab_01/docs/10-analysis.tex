\chapter{Аналитическая часть}

Расстояние Левенштейна между двумя строками - это минимальное количество операций, необходимых для превращения одной строки в другую.

Цены операций могут зависеть от вида операции (вставка, удаление, замена) и/или от участвующих в ней символов, отражая разную вероятность разных ошибок при вводе текста, и т. п. В общем случае:

\begin{itemize}
	\item $w(a,b)$ — цена замены символа $a$ на символ $b$.
	\item $w(\lambda,b)$ — цена вставки символа $b$.
	\item $w(a,\lambda)$ — цена удаления символа $a$.
\end{itemize}

Для решения задачи о редакционном расстоянии необходимо найти последовательность замен, минимизирующую суммарную цену. Расстояние Левенштейна является частным случаем этой задачи при

\begin{itemize}
	\item $w(a,a)=0$.
	\item $w(a,b)=1, \medspace a \neq b$.
	\item $w(\lambda,b)=1$.
	\item $w(a,\lambda)=1$.
\end{itemize}

\section{Рекурсивный алгоритм нахождения расстояния Левенштейна}

Расстояние Левенштейна между двумя строками $S_1$ и $S_2$ может быть вычислено по реккурентной формуле \ref{eq:D}, где $|S_1|$ означает длину строки $S_1$, $S_1[i]$ — i-ый символ строки $S_1$, функция $D_{S_1, S_2}(i, j)$ определена как:

\begin{equation}
\label{eq:D}
D_{S_1, S_2}(i, j) = \begin{cases}
0, &\text{$i = 0$, $j = 0$}\\
i, &\text{$j = 0$, $i > 0$}\\
j, &\text{$i = 0$, $j > 0$}\\
\min \lbrace\\
\qquad D_{S_1, S_2}(i, j-1) + 1\\
\qquad D_{S_1, S_2}(i-1, j) + 1 &\text{$i > 0, j > 0$}\\
\qquad D_{S_1, S_2}(i-1, j-1) + \theta(S_1[i], S_2[j])\\
\rbrace
\end{cases},
\end{equation}

а функция \ref{eq:m} определена как:
\begin{equation}
	\label{eq:m}
	\theta(a, b) = \begin{cases}
		0, &\text{если a = b}\\
		1, &\text{иначе}
	\end{cases}.
\end{equation}


Рекурсивный алгоритм реализует формулу \ref{eq:D}.
Функция $D$ составлена из следующих соображений:
\begin{enumerate}[label={\arabic*)}]
	\item для перевода пустой строки в пустую требуется ноль операций;
	\item для перевода пустой строки в непустую строку $S_1$ требуется $|S_1|$ операций;
	\item для перевода непустой строки $S_1$ в пустую требуется $|S_1|$ операций;
\end{enumerate}
Для перевода из строки $S_1$ в строку $S_2$ требуется выполнить последовательно некоторое количество операций (удаление, вставка, замена). Полагая, что $S_1', S_2'$  — строки $S_1$ и $S_2$ без последнего символа соответственно, цена преобразования из строки $S_1$ в строку $S_2$ может быть выражена как:
	\begin{enumerate}[label={\arabic*)}]
		\item сумма цены преобразования строки $S_1$ в $S_2'$  и цены проведения операции вставки, которая необходима для преобразования $S_2'$ в $S_2$;
		\item сумма цены преобразования строки $S_1'$ в $S_2$ и цены проведения операции удаления, которая необходима для преобразования $S_1$ в $S_1'$;
		\item сумма цены преобразования из $S_1'$ в $S_2'$ и операции замены, предполагая, что $S_1$ и $S_2$ оканчиваются на разные символы;
		\item цена преобразования из $S_1'$ в $S_2'$, предполагая, что $S_1$ и $S_2$ оканчиваются на один и тот же символ.
	\end{enumerate}
Минимальной ценой преобразования будет минимальное значение приведенных вариантов.

\section{Матричный алгоритм нахождения расстояния Левенштейна}

Прямая реализация формулы \ref{eq:D} может быть малоэффективна по времени исполнения при больших $i, j$, так как множество промежуточных значений $D(i, j)$ вычисляются заново несколько раз. 

Для оптимизации нахождения расстояния Левенштейна можно использовать матрицу в целях хранения соответствующих промежуточных значений. В таком случае алгоритм представляет собой построчное заполнение матрицы $M_{(|S_1|+1), (|S_2|+1)}$ значениями $D(i, j)$.


\section{Рекурсивный алгоритм нахождения расстояния Левенштейна с заполнением матрицы}

\label{sec:recmat}


Рекурсивный алгоритм заполнения можно оптимизировать по времени выполнения с использованием матричного алгоритма. Суть данного подхода заключается в использовании матрицы в качестве кэша при выполнении рекурсии, что позволяет снизить количество вычислений.


\section{Расстояние Дамерау-Левенштейна}

Расстояние Дамерау-Левенштейна может быть найдено по формуле \ref{eq:d}, которая задана как

\begin{equation}
\label{eq:d}
d_{a,b}(i, j) = \begin{cases}
\max(i, j), &\text{если }\min(i, j) = 0,\\
\min \lbrace \\
\qquad d_{S_1,S_1}(i, j-1) + 1,\\
\qquad d_{S_1,S_1}(i-1, j) + 1,\\
\qquad d_{S_1,S_1}(i-1, j-1) + \theta(a[i], b[j]), &\text{иначе}\\
\qquad \left[ \begin{array}{cc}d_{S_1,S_1}(i-2, j-2) + 1, &\text{если }i,j > 1;\\
\qquad &\text{}a[i] = b[j-1]; \\
\qquad &\text{}b[j] = a[i-1]\\
\qquad \infty, & \text{иначе}\end{array}\right.\\
\rbrace
\end{cases}
\end{equation}

Формула выводится по тем же соображениям, что и формула (\ref{eq:D}).
Как и в случае с рекурсивным методом, прямое применение этой формулы неэффективно по времени исполнения, то аналогично методу из \ref{sec:recmat} производится добавление матрицы для хранения промежуточных значений рекурсивной формулы.

\section{Вывод}
	В данном разделе были рассмотрены алгоритмы нахождения расстояния Левенштейна, а также Дамерау-Левенштейна, который учитывает возможность перестановки соседних символов. Формулы Левенштейна и Дамерау—Левенштейна для рассчета расстояния между строками задаются рекуррентно, а следовательно, алгоритмы могут быть реализованы рекурсивно или итерационно.
	
\clearpage
