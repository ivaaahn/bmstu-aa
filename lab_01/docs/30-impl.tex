\chapter{Технологическая часть}

\section{Требование к ПО}
К программе предъявляется ряд требований:
\begin{enumerate}
    \item предоставление возможность выбора алгоритма:
	\item возможность ввода строк в любой раскладке и регистре;
	\item печать на экран полученного расстояние и дополнительной информации (вспомогательные матрицы, затраченное время и объем памяти);
\end{enumerate}

\section{Средства реализации}
Для реализации программы нахождения расстояние Левенштейна был выбран язык программирования Python3\cite{python}. Данный выбор обусловлен простотой и скоростью написания программ, а также наличием встроенных библиотек для построения графиков функций и тестирования. В качестве среды разработки был выбран \texttt{PyCharm}\cite{pycharm}, как наиболее популярная IDE для языка Python3. 

\section{Реализация алгоритмов}

В листингах 3.1 - 3.4 приведена реализация алгоритмов нахождения расстояния Левенштейна и Дамерау-Левенштейна.

\begin{lstlisting}[label=LevRec,caption=Функция нахождения расстояния Левенштейна рекурсивно,language=python]
def levenshtein_recurs(s1: str, s2: str, depth: int = 0) -> tuple[int, int]:
    if not (s1 and s2):
        return (len(s1) + len(s2)), depth

    if s1[-1] == s2[-1]:
        return levenshtein_recurs(s1[:-1], s2[:-1], depth + 1)

    dist, depth = min(
        levenshtein_recurs(s1[:-1], s2, depth + 1),
        levenshtein_recurs(s1, s2[:-1], depth + 1),
        levenshtein_recurs(s1[:-1], s2[:-1], depth + 1),
    )
    return dist + 1, depth
\end{lstlisting}

\begin{lstlisting}[label=LevRecMem,caption=Функция нахождения расстояние Левенштейна рекурсивно с матрицей в качестве кэша,language=python]
def levenshtein_recurs_mem(s1: str, s2: str, depth: int = 0) -> tuple[int, int, CacheMatrix]:
    _CACHE: CacheMatrix = [[None for _ in range(len(s2) + 1)] for _ in range(len(s1) + 1)]
    result = _lev_rec_mem_inner(s1, s2, _CACHE, depth)
    return result[0], result[1], _CACHE


def _lev_rec_mem_inner(s1: str, s2: str, m: CacheMatrix, depth: int = 0) -> tuple[int, int]:
    ls1, ls2 = len(s1), len(s2)

    if m[ls1][ls2] is None:
        if not (s1 and s2):
            m[ls1][ls2] = ls1 + ls2
        elif s1[-1] == s2[-1]:
            m[ls1][ls2], depth = _lev_rec_mem_inner(s1[:-1], s2[:-1], m, depth + 1)
        else:
            dist, depth = min(_lev_rec_mem_inner(s1[:-1], s2, m, depth + 1),
                              _lev_rec_mem_inner(s1, s2[:-1], m, depth + 1),
                              _lev_rec_mem_inner(s1[:-1], s2[:-1], m, depth + 1))
            m[ls1][ls2] = dist + 1

    return m[ls1][ls2], depth
\end{lstlisting}

% \vspace{2\baselineskip}

\begin{lstlisting}[label=DamLev,caption=Функция нахождения расстояния Дамерау-Левенштейна матрично,language=python]
def damerau_levenshtein(s1: str, s2: str) -> tuple[int, CacheMatrix]:
    ls1, ls2 = len(s1), len(s2)
    m = [[(i + j) if i * j == 0 else 0 for j in range(ls2 + 1)] for i in range(ls1 + 1)]

    for i in range(1, ls1 + 1):
        for j in range(1, ls2 + 1):
            if s1[i - 1] == s2[j - 1]:
                m[i][j] = m[i - 1][j - 1]
            else:
                m[i][j] = 1 + min(
                    m[i - 1][j],
                    m[i][j - 1],
                    m[i - 1][j - 1],
                    m[i - 2][j - 2]
                    if all((i >= 2, j >= 2, s1[i - 1] == s2[j - 2], s1[i - 2] == s2[j - 1])) else inf
                )
    return m[-1][-1], m
\end{lstlisting}


\begin{lstlisting}[label=DamLevRec,caption=Функция нахождения расстояния Дамерау-Левенштейна рекурсивно,language=python]
def damerau_levenshtein_recursive(s1: str, s2: str, depth: int = 0) -> tuple[int, int]:
    if not (s1 and s2):
        return len(s1) + len(s2), depth
    if s1[-1] == s2[-1]:
        return damerau_levenshtein_recursive(s1[:-1], s2[:-1], depth + 1)
    ls1, ls2 = len(s1), len(s2)
    dist, depth = min(
        damerau_levenshtein_recursive(s1[:-1], s2, depth + 1),
        damerau_levenshtein_recursive(s1, s2[:-1], depth + 1),
        damerau_levenshtein_recursive(s1[:-1], s2[:-1], depth + 1),
        damerau_levenshtein_recursive(s1[:-2], s2[:-2], depth + 1)
        if all((ls1 >= 2, ls2 >= 2, s1[-1] == s2[-2], s1[-2] == s2[-1])) else inf,
    )
    return dist + 1, depth
\end{lstlisting}

\section{Тестовые данные}

В таблице \ref{tabular:functional_test} приведены функциональные тесты для алгоритмов вычисления расстояния Левенштейна и Дамерау — Левенштейна. В колонке "Ожидаемый результат"\ указаны два числа - ожидаемые результаты работы алгоритмов Левенштейна и Дамерау-Левенштейна соответственно. Тестирование проводилось при помощи модуля \texttt{pytest}\cite{pytest} Все тесты пройдены успешно.

\begin{table}[ht]
	\begin{center}
		\caption{\label{tabular:functional_test} Функциональные тесты}
		\begin{tabular}{|c|c|c|c|} 
			\hline
			\bfseries Класс & \bfseries Строка 1  & \bfseries Строка 2 &  \bfseries Ожидаемый результат \\
			\hline
			Пустые строки & &  & 0 0 \\
			\hline
			Одна из строк пустая & & Бауманка & 8 8\\
            \hline
            Одинаковые строки &Бауманка & Бауманка & 0 0\\
            \hline
            Строки одинаковой длины &Бауманка & Бвуманкк & 2 2\\
            \hline
            Транспозиция&Бауманка & Бауманак & 2 1\\
            \hline
            Двойная тразпозиция & Александр & Аелксанрд & 4 2 \\
            \hline
            Полностью разные строки&МГТУ & армия & 5 5 \\
            \hline
            Добавить один символ&МГТ & МГТУ & 1 1 \\
            \hline
            Удалить один символ&МГТУ & МГТ & 1 1 \\
            \hline
            Заменить один символ&МГТУ & МВТУ & 1 1 \\
			\hline
			Латиница&Hello & Hi & 4 4 \\
			\hline
		\end{tabular}
	\end{center}
\end{table}

\section{Вывод}
В данном разделе были разработаны исходные коды четырех алгоритмов: \begin{enumerate}
  	\item вычисление расстояния Левенштейна рекурсивно;
  	\item вычисление расстояния Левенштейна рекурсивно с использованием матрицы в качестве кэша;
	\item вычисление расстояния Дамерау—Левенштейна рекурсивно;
	\item вычисление расстояния Дамерау—Левенштейна с заполнением матрицы;
\end{enumerate}
