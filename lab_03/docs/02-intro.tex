\chapter*{Введение}
\addcontentsline{toc}{chapter}{Введение}

Алгоритм сортировки — это алгоритм для упорядочивания эле­ментов в массиве каких-либо 
данных. Другими словами, это перегруппи­ровка данных по какому-либо ключу. 

В основном алгоритмы сортировки являются не конечной целью, а промежуточной - 
отсортированные данные намного легче обрабаты­вать. Например, чтобы удалить 
дубликаты в массиве, будет разумносначала отсортировать массив. Также намного 
легче производить поиск какой-либо информации в заранее отсортированном массиве. 
Ассимпто­тическая сложность при поиске в отсортированном массиве составляет
$O(\log n)$ против $O(n)$ в произвольном массиве данных.

Каждый алгоритм сортировки имеет свои особенности реализа­ции, однако в конечном итоге
эту задачу можно разбить на следующие шаги:
\begin{enumerate}
	\item Сравнение пары элементов массива. Для этого нужно указать	способ сравнения 
	(бинарное отношение) для двух элементов, например, учесть природу данных или задать 
	ключ в случае, если записи в массиве имеют несколько полей;
	\item Перестановка двух элементов, которые могут располагаться как последовательно 
	друг за другом, так и находиться в разных частях мас­сива;
	\item Алгоритм должен завершать свою работу, когда массив становит­ся полностью 
	упорядоченным.
\end{enumerate}

Алгоритмы сортировки оцениваются по скорости выполнения и эффективности использования памяти.
Обычно асимптотическая сложность алгоритмов сортировок массивов размера $n$ лежит в 
диапазоне от $O(n)$ до $O(n^2)$


Целью работы явялется анализ производительности алгоритмов сортировки. Для достижения поставленной 
цели необходимо выполнить следующие задачи:
\begin{itemize}
	\item рассмотреть алгоритмы сортировки пузырьком, вставками и выбором;
	\item сравнить алгоритмы, выявить их достоинства и недостатки;
	\item разработать схемы выбранных алгоритмов;
	\item определить модель вычислений трудоемкости;
	\item на основе теоретических расчетов и выбранной модели вычислений 
	провести cравнительный анализ трудоёмкости алгоритмов;
	\item определить средства для реализации алгоритмов;
	\item реализовать данные алгоритмы сортировки;
	\item выделить классы эквивалентности для сортировки массива;
	\item на основе выделенных классов эквивалентности разработать функциональные тесты;
	\item экспериментально провести сравнительный анализ производительности алгоритмов.
\end{itemize}




