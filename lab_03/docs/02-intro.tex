\chapter*{Введение}
\addcontentsline{toc}{chapter}{Введение}

Алгоритм сортировки — это алгоритм для упорядочивания эле­
ментов в массиве каких-либо данных. Другими словами, это перегруппи­
ровка данных по какому-либо ключу. По некоторым источникам, именно
программа сортировки стала первой программой для вычислительных
машин.

В основном алгоритмы сортировки являются не конечной целью,
а промежуточной - отсортированные данные намного легче обрабаты­
вать. Например, чтобы удалить дупликаты в массиве, будет разумно
сначала отсортировать массив. Также намного легче производить поиск
какой-либо информации в заранее отсортированном массиве. Ассимпто­
тическая сложность при поиске в отсортированном массиве составляет
O(log(n)) против O(n) в произвольном массиве данных.

Каждый алгоритм сортировки имеет свои особенности реализа­
ции, однако в конечном итоге эту задачу можно разбить на следующие
шаги:
\begin{enumerate}
	\item Сравнение пары элементов массива. Для этого нужно указать	способ сравнения (бинарное отношение)
	 для двух элементов, например,	учесть природу данных или задать ключ в случае, если записи в массиве
	  имеют несколько полей;
	\item Перестановка двух элементов, которые могут располагаться как последовательно друг за другом, 
	так и находиться в разных частях мас­сива;
	\item Алгоритм должен завершать свою работу, когда массив становит­ся полностью упорядоченным.
\end{enumerate}


Алгоритмы сортировки оцениваются по скорости выполнения и эффективности использования памяти.
Обычно асимптотическая сложность алгоритмов сортировок массивов размерностью $n$ лежит в 
диапазоне от $O(n)$ до $O(n^2)$


Целью работы явялется анализ эффективности алгоритмов сортировки. Для достижения поставленной цели необходимо
выполнить следующие задачи:
Для достижения поставленной цели необходимо выполнить следующие задачи:
\begin{itemize}
	\item реализовать три алгоритма сортировки: пузрьком, вставками и выбором;
	\item определить модель вычислений трудоемкости;
	\item на основе теоретических расчетов и выбранной модели вычислений провести cравнительный анализ трудоёмкости алгоритмов;
	\item экспериментально провести сравнительный анализ производительности алгоритмов.
\end{itemize}




