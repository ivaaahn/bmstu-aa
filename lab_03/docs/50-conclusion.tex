\chapter*{Заключение}
\addcontentsline{toc}{chapter}{Заключение}

В ходе выполнения лабораторной работы была выполнены поставленные задачи, а именно:

\begin{itemize}
	\item рассмотрены алгоритмы сортировки пузырьком, вставками и выбором;
	\item разработаны схемы выбранных алгоритмов;
	\item на основе теоретических расчетов и выбранной модели вычислений 
	проведен cравнительный анализ трудоёмкости алгоритмов;
	\item реализованы выбранные алгоритмы сортировки;
	\item разработаны функциональные тесты для программы;
	\item экспериментально проведен сравнительный анализ быстродействия алгоритмов.
\end{itemize}


Экспериментально были установлены различия в производительности сортировок пузырьком, выбором
и вставками. Так алгоритмы сортировки выбором и пузырьком на отсортированных массивах работают 
медленнее, чем сортировка вставками в 3 и 4 раза соответственно при 10 элементах, и в 300 и 430 раз
соответственно при 1000 элементах.

На обратно отсортированных массивах сортировка вставками сравнима по быстродействию с сортировкой выбором.
При длине массива 500 элементов разница в быстродействии 4\%, а при 1000 - всего лишь в 1\%. При этом 
сортировка пузырьком медленнее остальных примерно в 2.2 раза.

Однако на практике массивы никак не упорядочены. В таких случаях сортировка вставками снова оказывается 
быстрее остальных. Она быстрее сортировки выбором и пузырьком в 1.2 и 2 раза соответственно при 10 
элементах, в 1.9 и 3.6 раза - при 500 элементах, в 1.8 и 3.5 раза - при 1000 элементах.

Следовательно, можно сделать вывод, что сортировка вставками является наиболее эффективной сортировкой 
из рассмотренных, а сортировка пузырьком, наоборот, наименее эффективной.
