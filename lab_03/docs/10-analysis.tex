\chapter{Аналитическая часть}

В данном разделе будут описаны ключевые идеи каждого из вы­бранных алгоритмов сортировки[1].

\section{Алгоритмы}

Сортировка пузырьком является одним из простейших алгорит­мов сортировки данных.
Пусть имеется массив для сортировки размерностью $N$ .
В классической версии массив обходится слева направо. Алгоритм
состоит из повторяющихся проходов по этому массиву - за каждый про­
ход элементы последовательно сравниваются попарно и, если порядок
в паре неверный, выполняется перестановка элементов так, чтобы пара
стала упорядоченной.
Проходы по массиву повторяются 𝑁 − 1 раз или до тех пор, пока
на очередном проходе не окажется, что обмены больше не нужны, что
означает — массив отсортирован.
В результате каждого прохода по внутреннему циклу определяет­
ся очередной максимум подмассива, этот максимум помещается в конец
рядом с предыдущим наибольшим элементом, вычисленным на преды­
дущем проходе, а наименьший элемент перемещается ближе к началу
массива («всплывает» до нужной позиции, как пузырёк в воде — отсюда
и название алгоритма).

\section{Алгоритм Винограда для перемножения матриц}

Если рассмотреть результат умножения двух матриц, то видно, что каждый элемент в нем представляет собой скалярное произведение соответствующих строки и столбца исходных матриц.
Можно заметить также, что такое умножение допускает предварительную обработку, позволяющую часть работы выполнить заранее  \cite{win-book}. 

Рассмотрим два вектора $V = (v_1, v_2, v_3, v_4)$ и $W = (w_1, w_2, w_3, w_4)$. Их скалярное произведение равно:
\begin{equation}
	\label{eq:defmul}
	V \cdot W = v_1w_1 + v_2w_2 + v_3w_3 + v_4w_4
\end{equation}
что эквивалентно
\begin{equation}
	\label{eq:newmul}
		V \cdot W = (v_1 + w_2)(v_2 + w_1) + (v_3 + w_4)(v_4 + w_3) - v_1v_2 - v_3v_4 - w_1w_2 - w_3w_4.
\end{equation}

На первый взгляд неочевидно, в чем преимущество выражения \ref{eq:newmul}, поскольку оно требует вычисления большего 
количества операций, чем классическое \ref{eq:defmul}:
вместо четырёх умножений -- шесть, а вместо трёх сложений -- десять. Заметим, что выражение в правой части формулы \ref{eq:newmul}
допускает предварительную обработку: его части можно вычислить заранее и запомнить для каждой строки первой матрицы
и для каждого столбца второй, что позволит для каждого элемента выполнять лишь два умножения и пять сложений.
Из-за того, что операция сложения быстрее операции умножения в ЭВМ, на практике алгоритм должен работать быстрее стандартного.


\section{Вывод}

В данном разделе был рассмотрены классический алгоритм перемножения матриц и алгоритм Винограда, позволяющий сократить количество операций умножения, 
а также использовать предварительные рассчеты.