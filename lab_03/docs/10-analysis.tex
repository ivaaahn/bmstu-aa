\chapter{Аналитическая часть}

В данном разделе будут описаны ключевые идеи каждого из вы­бранных алгоритмов сортировки.

\section{Сортировка пузырьком}

Сортировка пузырьком является одним из простейших алгорит­мов сортировки данных.

Пусть имеется массив для сортировки размерностью $N$. В классической версии массив 
обходится слева направо. Алгоритм состоит из повторяющихся проходов по этому
массиву - за каждый про­ход элементы последовательно сравниваются попарно и, 
если порядок в паре неверный, выполняется перестановка элементов так, чтобы пара
стала упорядоченной.\cite{bubble}

Проходы по массиву повторяются $N-1$ раз или до тех пор, пока на очередном проходе
 не окажется, что обмены больше не нужны, что означает — массив отсортирован.
В результате каждого прохода по внутреннему циклу определяет­ ся очередной максимум 
подмассива, этот максимум помещается в конец рядом с предыдущим наибольшим 
элементом, вычисленным на преды­ дущем проходе, а наименьший элемент 
 ближе к началу массива («всплывает» до нужной позиции, как пузырёк в 
 воде — отсюда и название алгоритма).


\section{Сортировка выбором}

Сортировка выбором --- это, возможно, самый простой в реализации алгоритм сортировки. 
Как и в большинстве других подобных алгоритмов, в его основе лежит операция сравнения.
Сравнивая каждый элемент с каждым, и в случае необходимости производя обмен, метод 
приводит последовательность к необходимому упорядоченному виду.

Шаги алгоритма\cite{selandins}:

\begin{enumerate}
	\item Находится номер минимального значения в текущем списке.
	\item Производится обмен этого значения со знаением первой 
    неотсортированной позиции (обмен не нужен, если минимальный элемент 
    уже находится на данной позиции).
    \item Сортируется хвост списка, исключив из рассмотрения уже отсортированные элементы.
\end{enumerate}

\section{Сортировка вставками}

Идеи алгоритма сортировки вставками следующие\cite{selandins}:
\begin{enumerate}
	\item Массив делится на две части --- отсортированную и неотсортированную.
	\item Из неотсортированной части извлекается любой элемент.
	\item Элемент вставляется в отсортированную часть массива, сохраняя ее упорядоченность.
	\item Так происходит до тех пор, пока массив не будет полностью отсортирован.
\end{enumerate}

Поскольку первая часть массива всегда отсортирована, в ней достаточно быстро 
можно найти своё место для элемента, извлеченного из неотсортированной части.

Метод выбора очередного элемента из исходного массива произволен, однако обычно
 (и с целью получения устойчивого алгоритма сортировки), элементы вставляются по 
 порядку их появления во входном массиве.

\section{Вывод}

В данном разделе были рассмотрены ключевые идеи выбранных алгоритмов
сортировки данных: пузырьком, выбором и вставками. 