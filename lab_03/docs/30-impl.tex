\chapter{Технологическая часть}
В данном разделе приведны требования к программному обеспечению, средства реализации и листинги кода
\section{Требования к ПО}
К программе предъявляется следующие требования:
\begin{itemize}
    \item предоставить возможность работы в двух режимах: проведения эксперимента и ручного тестирования;
    \item в режиме ручного тестирования предоставить пользователю как самостоятельно осуществлять ввод двух 
    массива, который необходимо отсортировать, так и использовать автозаполнение псевдослучайными числами от 0 до 99;
	\item результатом работы программы в ручном режиме должны являвляться три массива - результаты сортировки исходного
    массива тремя сортировами: пузырьком, выбором и вставками;
    \item в режиме проведения эксперимента результатом работы программы является результат эксперимента в текстовом виде,
    а также построенные графики в виде файла с изображением;
    \item программа должна корректно обрабатывать любые действия пользователя, в том числе, например, ввод массива длиной 0.
\end{itemize}

\section{Средства реализации}
Для реализации алгоритмов перемножения матриц был выбран язык программирования Python3\cite{python}, что обусловлено простотой 
и скоростью написания программ, а также наличием встроенных библиотек для построения графиков функций и тестирования. 
В качестве среды разработки был выбран \texttt{PyCharm}\cite{pycharm}, как наиболее популярная IDE для Python3. 

\section{Функциональное тестирование}

При разработке функционалых тестов были выделены следующие классы эквивалентности:

\begin{itemize}
    \item отсортированный массив;
    \item массив, отсортированный в обратном порядке;
    \item массив из нечетного количества элементов;
    \item массив из четного количество элементов;
    \item массив из одного элемента;
    \item пустой массив;
    \item произвольный массив.
\end{itemize}

В соответствии с данными классами эквивалентности были разработаны тесты, представленные в таблице \ref{tab:tests}.
\begin{table}[h!]
	\begin{center}
        \captionof{table}{Тестирование функций}
		\begin{tabular}{|c|c|c|}
			\hline
			Входные данные & Ожидаемый результат & Реальный результат \\ 
            \hline
            $[-1, 0, 5, 9, 11]$ &
            $[-1, 0, 5, 9, 11]$ &
			$[-1, 0, 5, 9, 11]$ \\
            \hline
            $[11, 9, 5, 0, -1]$ &
            $[-1, 0, 5, 9, 11]$ &
			$[-1, 0, 5, 9, 11]$ \\
            \hline
            $[3, -1, 0, 7, 5]$ &
            $[-1, 0, 3, 5, 7]$ &
            $[-1, 0, 3, 5, 7]$ \\
            \hline
            $[3, -1, 0, 7]$ &
            $[-1, 0, 3, 7]$ &
            $[-1, 0, 3, 7]$ \\
            \hline
            $[7]$ &
            $[7]$ &
            $[7]$ \\
            \hline
            $[ ]$ &
            $[ ]$ &
            $[ ]$ \\
            \hline
        \end{tabular}
        \label{tab:tests}
	\end{center}
\end{table}

Все тесты пройдены успешно.

\section{Реализация алгоритмов}

В листинге \ref{lst:bubble} представлена реализация алгоритма сортировки пузырьком.
\begin{lstlisting}[style=mypython,label=lst:bubble,caption={Реализация алгоритма сортировки пузырьком},language=python]
def bubble_sort(a: ArrayInt, n: int) -> None:
    i = 0
    while i < n - 1:

        j = 0
        while j < n - i - 1:
            if a[j] > a[j + 1]:
                a[j], a[j + 1] = a[j + 1], a[j]
            j += 1

        i += 1
\end{lstlisting}

В листинге \ref{lst:insert} представлена реализация алгоритма сортировки вставками.
\begin{lstlisting}[style=mypython,label=lst:insert,caption={Реализация алгоритма сортировки вставками},language=python]
def insertion_sort(a: ArrayInt, n: int) -> None:
    i = 1
    while i < n:
        key = a[i]

        j = i - 1
        while j >= 0 and key < a[j]:
            a[j + 1] = a[j]
            j -= 1
        a[j + 1] = key

        i += 1
\end{lstlisting}

В листинге \ref{lst:select} представлена реализация алгоритма сортировки выбором.
\begin{lstlisting}[style=mypython,label=lst:select,caption={Реализация алгоритма сортировки выбором},language=python]
def selection_sort(a: ArrayInt, n: int) -> None:
    i = 0
    while i < n - 1:
        min_idx = i

        j = i + 1
        while j < n:
            if a[j] < a[min_idx]:
                min_idx = j
            j += 1

        a[min_idx], a[i] = a[min_idx], a[i]

        i += 1
\end{lstlisting}


\section{Вывод}
В данном разделе были выделены классы эквивалентности для операции сортировки массива, 
на основе которых разработаны функциональные тесты для ПО, также были реализованы сами алгоритмы 
сортировки массива на языке Python3.